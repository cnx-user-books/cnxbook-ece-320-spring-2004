%AUTHOR: Michael Kramer
One of MATLAB's features is its ability to quickly create a visual interface
that lets you use standard GUI controls (such as sliders, checkboxes and
radio buttons) to call MATLAB scripts. An example of such an interface is 
available in the \verb+V:\MAT_INT+ directory, which contains two files:
\begin{itemize}
  \item{\verb+ser_set.m+: Initializes the serial port and user interface}
  \item{\verb+wrt_slid.m+: Called when sliders are moved to send new data}
\end{itemize}

\section{Creating a MATLAB user interface}

The following code (\verb+ser_set.m+) initializes the serial port COM2, then
creates a minimal user interface consisting of three sliders.

\setlength{\baselineskip}{0.4cm}
\listinginput{1}{ser_set.m}
\setlength{\baselineskip}{0.5cm}

Line 4 of this code uses the Windows NT \verb+mode+ command to set up
COM port 2 (which is connected to the DSP) to match the serial port 
settings on the DSP evaluation board: 38,400 bps, no parity, 8 data bits
 and 1 stop bit. Line 7 then creates a new MATLAB figure for the controls;
this prevents the controls from being overlaid on to of any graph you
may have already created. 

Lines 12 through the end create the three sliders for the user interface. 
Several parameters are used to specify the behavior of each slider. The
first parameter, \verb+Fig+, tells the slider to create itself in the 
window we created in Line 7. The rest of the parameters are property/value
pairs:

\begin{itemize}
  \item{\verb+units+: \verb+Normal+ tells MATLAB to use positioning relative 
        to the window boundaries.}
  \item{\verb+pos+: Tells MATLAB where to place the control.}
  \item{\verb+style+: Tells MATLAB what type of control to place. \verb+slider+
        creates a slider control.}
  \item{\verb+value+: Tells MATLAB the default value for the control.}
  \item{\verb+max+: Tells MATLAB the maximum value for the control.}
  \item{\verb+min+: Tells MATLAB the minimum value for the control.}
  \item{\verb+callback+: Tells MATLAB what script to call when the control is 
        manipulated. \verb+wrt_slid+ is a MATLAB file that writes the values of
        the controls to the serial port.}
\end{itemize}

\subsection{The User Interface Callback Function}

Every time a slider is moved, the \verb+wrt_slid.m+ file is called:

\setlength{\baselineskip}{0.4cm}
\listinginput{1}{wrt_slid.m}
\setlength{\baselineskip}{0.5cm}

Line 4 of \verb+wrt_slid.m+ opens COM2 for writing. (It's already
been initialized.) Then Line 7 retrieves the value from the slider
using MATLAB's \verb+get+ function to retrieve the \verb+value+
property. The value is then rounded off to create an integer,
and the integer is sent as an 8-bit quantity to the DSP in Line 8.
(The number that is sent at this step will appear when the serial
port is read with \verb+READ_SER+ in your code.) Then the other
two sliders are sent in the same way.

Line 17 sends 0xFF (255) to the DSP, which can be used to indicate
that the three previously-transmitted values represent a complete set
of data points. This can be used to prevent the DSP and MATLAB from
losing synchronization if a transmitted character is not received by
the DSP.

Line 20 closes the serial port. Note that MATLAB buffers the data
being transmitted, and data is often not sent until the serial port
is closed. Make sure you close the port after sending a data block
to the DSP.