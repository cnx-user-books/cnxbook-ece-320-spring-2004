%AUTHOR: Swaroop Appadwedula
\paragraph{Exercise}
Take a simple signal (e.g. one period of a sinusoid at some frequency) 
and plot its autocorrelation sequence for appropriate values of $l$.  
You may wish to use the \verb+xcorr+ \matlab function to
compare with your own version of this function.  At what time
shift $l$ is $r_{ss}(l)$ maximized and why?  Is there any
symmetry in $r_{ss}(l)$?  What does $r_{ss}(l)$ look like for
periodic signals?

\paragraph{Exercise}
Write your own version of the Levinson-Durbin algorithm in \matlab.  
Note that \matlab uses indexing from $1$ rather than $0$.  A good idea 
would be to start the loop with $i=2$, and appropriately shift 
the variables $k, E, \alpha$, and $r_{ss}$ to start at $i=1$ and 
$j=1$. Be careful with indices such as $i-j$, since these could 
still be $0$.  

Apply your algorithm to a $20-30ms$ segment of a speech signal.  
Use the small powered microphone or professional microphone
available in the DSP lab to record \verb+.wav+ audio files on the
PC using the application Sound Recorder.  Typically, a
sampling rate of $8kHz$ is a good choice
for voice signals which have maximum
frequency below $4kHz$.  
You will use these audio files to test algorithms in  
\matlab.  The following functions will help you
read, write and play audio files in \matlab:
\verb+wavread,wavwrite,sound+.  You can also convert CD
tracks to a \verb+.wav+ file using the Mediaplayer application
on the PC.

The output of the algorithm is the prediction coefficients $a_k$ 
(usually about $P=10$ coefficients is sufficient), 
which represent the speech segment containing significantly 
more samples.  The LPC coefficients are a compressed representation 
of the ordinal speech segment.  Compare the 
coefficients generated by your function with those generated 
by the \verb+levinson+ or \verb+lpc+ functions available in the 
\matlab toolbox.  Next, plot the frequency response of the IIR model 
represented by the LPC coefficients where 
$a_{k} = \alpha_{k,P}, k=1,2\ldots,P$ (see (\ref{equ:IIR})).  
What is the fundamental frequency of the speech segment?  Is 
there any similarity in the prediction coefficients for 
different $20-30ms$ segments of the same vowel sound? 
How could the prediction coefficients be used for recognition?

