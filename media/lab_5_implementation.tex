In this lab you are to implement and optimize
the frequency shift keying (FSK) digital transmitter
and pseudo-noise (PN) sequence generator shown in Figure~\ref{fig:trans}.
For the lab grade, you will
be judged on the execution time of your system (memory usage need not
be minimized).

\section{Implementation}
You will implement and optimize the complete system shown in
Figure~\ref{fig:trans}
over the next two weeks.  You may write in C, assembly, or 
any combination of the two; choose whatever will allow you to 
write the fastest code.  The optimization process will probably 
be much easier if you plan for optimization before you begin 
any programming.

\subsection{PN Generator}
Once you have planned your program strategy, implement the PN 
generator from Figure~\ref{fig:pn-gen} and verify that it is working.  
If you are programming in assembly, you may wish to refer 
to the description of assembly instructions for logical operations
in Section 2-2 of the C54x Mnemonic Instruction Set reference.  
Initialize the shift register to one.  

In testing the PN generator, you may find the file 
\verb+v:\ece320\54x\dspclib\pn_output.mat+ helpful.  
To use it, type \verb+load v:\ece320\54x\dspclib\pn_output+
at the \matlab command prompt.  This will load a vector 
\verb+pn_output+ into memory.  The vector contains 500 elements, 
which are the first 500 output bits of the PN generator.
Be prepared to prove to
a TA that your PN generator works properly as part of your quiz.

\subsection{Transmitter}
For your transmitter implementation you are to use
the data-block-to-carrier-frequency mapping in the table 
on page~\pageref{table}
and a digital symbol period of $T_{symb} = 32$ samples.  

Viewing the transmitted signal on the oscilloscope may help 
you determine whether your code works properly, but you should 
check it more carefully by setting breakpoints in Code Composer and 
using the \verb+Memory+ option from the \verb+View+ menu to view 
the contents of memory.  The vector signal analyzer (VSA) provides 
another method of testing. 

\subsubsection{Testing with the VSA}

The VSA is an instrument capable of 
demodulating digital signals. You may use  
the VSA to demodulate your FSK signal and display the 
symbols received.

\paragraph{Configuring the VSA:}  
The VSA is the big HP unit on a cart in the front of the
classroom. Plug the output from the DSP board into the ``Channel 1'' jack
on the front of the vector signal analyzer, and then turn on the
analyzer and follow these instructions to display your output:

After powering the signal analyzer up, the display will not be in the
correct mode. Use the following sequence of keypresses to set it up
properly:\footnote{If this doesn't work, hit ``Save/Recall,'' F7 (Catalog), 
point at ``ECE320.STA'' with the wheel, and hit F5 (Recall State) and
F1 (Enter).}

\begin{itemize}
\item ``Freq'' button, followed by F1 (center), 11.025 (on the keypad), and F3 (KHz) 
\item F2 (span), 22, and F3 (KHz) 
\item ``Range,'' then F5 (ch1 autorange up/down)
\item ``Instrument Mode,'' then F3 (demodulation) 
\end{itemize}

\paragraph{Viewing the signal spectrum on the VSA: }
The VSA is also capable of displaying the spectrum of a signal.  
Hook up the output of your PN generator to the VSA and set it up 
properly to view the spectrum of the random sequence.  Hit ``Instrument
Mode'' and then F1 (Scalar) to see the spectrum.  Note that you
can also use your Lab 4 code for this purpose.

Does what you see match the \matlab simulations?

\subsection{Optimization}
One purpose of this lab is to teach optimization and efficient
code techniques. For this reason, for your lab grade 
\emph{you will be judged primarily on the total execution
time of your system.}  You are not required to optimize memory 
use.  Note that by execution time we mean
cycle count, not the number of instructions in your program.
Remember that several of the TMS320C54xx instructions take
more than one cycle. The multicycle instructions are primarily
the multi-word instructions, including instructions that
take immediates, like \verb+stm+, and instructions using 
direct addressing of memory (such as \verb+ld *(temp),A+).
Branch and repeat statements also require several cycles
to execute.  Most C instructions take more than one cycle.
The debugger can be used to
determine the exact number of cycles used by your code; 
ask your TA to demonstrate. 
However, since the number of
execution cycles used by an instruction is usually determined
by the number of words in its encoding, the easiest way to estimate 
the number of cycles used by your code is to count the number
of instruction words in the \verb+.lst+ file or the disassembly
window in the debugger.

We will grade you based on the number of cycles used between
the return from the WAITDATA call and the arrival at the next
WAITDATA call in assembly, or the return from one WaitAudio call 
and the arrival at the next WaitAudio call in C. If the number of 
cycles between the two points  is variable, the maximum possible 
number of cycles will be counted. You must use the \verb+core.asm+ file 
in \verb+v:\ece320\54x\dsplib\core.asm+ or 
the C core file in  \verb+v:\ece320\54x\dspclib\core.asm+ as provided 
by the TAs;  
{\bf these files may not be modified}. You explicitly may not change the 
number of samples read and written by each WAITDATA or WaitAudio call! 
We reserve the right to test your code by substituting the test vector
core file.

The student or students with the fastest code (as defined
by the grading criteria above) will enjoy free pizza at 
Papa Del's in the company of Professor Bresler and the TAs.


\section{Grading}

This is a two-week lab. Your prelab is due a week after the
quiz for Lab 4, and
the quizzing occurs two weeks after the quiz for Lab 4. 

Grading for this lab will be a bit different from past labs:

\begin{itemize}
\item 1 point: Prelab
\item 2 points: Working code, implemented from scratch in 
	assembly language or C. 
\item 5 points: Optimization. These points will be assigned
	based on your cycle counts and the optimizations
 	you have made. 
\item 2 points: Oral quiz. 
\end{itemize}

Your final assembly and/or C source code for the pizza 
competition must be emailed to ece320@ews.uiuc.edu no 
later than 11:59 PM on Friday, March 15.  
However, your
optimization grade will be assigned based on the 
code turned in during your assigned lab section,
and is subject to the usual policies for late code.
