%AUTHOR: Michael Kramer
The FFT provided requires that the input be in bit-reversed
order, with alternating real and imaginary components.
Bit-reversed
addressing is a convenient way to order input $x[n]$
into a FFT so that the output $X(k)$ is in sequential order
(i.e. $X(0), X(1),\ldots X(N-1)$ for an $N$-point FFT).
The following table illustrates the bit-reversed order for
an eight-point sequence.
\begin{center}
\begin{tabular}{|c|c|c|c|} \hline
Input Order & Binary Representation & Bit-Reversed Representation
& Output Order \\ \hline
0 & 000 & 000 & 0 \\
1 & 001 & 100 & 4 \\
2 & 010 & 010 & 2 \\
3 & 011 & 110 & 6 \\
4 & 100 & 001 & 1 \\
5 & 101 & 101 & 5 \\
6 & 110 & 011 & 3 \\
7 & 111 & 111 & 7 \\
\hline
\end{tabular}
\end{center}

The following routine performs the bit-reversed
reordering of the input data.  
The routine assumes that the input is stored in data memory
starting at the location labeled \verb+_bit_rev_data+, which must 
be aligned to the least power of two greater than the input buffer 
length, and consists of 
alternating real and imaginary parts.  
Because our input data is going to be purely real in this
lab, you will have to make sure that you set the imaginary
parts to zero by zeroing out every other memory location.

\setlength{\baselineskip}{0.39cm}
\setlength{\parskip}{0.4cm}
\listinginput{1}{bit_rev.asm}
\setlength{\baselineskip}{0.5cm}
\setlength{\parskip}{0.5cm}

As mentioned, in the above code \verb+_bit_rev_data+ is a label indicating
the start of the input data and \verb+_fft_data+ is a label indicating 
the start of a circular buffer where the bit-reversed data will be written.
Note that although \verb+AR7+ is not
used by the bit-reversed routine directly, it is used extensively
in the FFT routine to keep track of the start of the FFT data space.

In general, to have a pointer index memory in bit-reversed order, the
\verb+AR0+ register needs to be set to one-half the length of
the circular buffer; a statement such as \verb!ARx+0B! is
used to move the \verb+ARx+ pointer to the next location.
For more information regarding the bit-reversed
addressing mode, refer to page 5-18 in the {\em TI-54x CPU and Peripherals}
manual.  Is it possible to bit-reverse a buffer in place?  For 
a diagram of the ordering of the data expected by the FFT routine, 
see Figure 4-10 in the {\em TI-54x Applications Guide}.  Note that the 
FFT code uses all the pointers available and does not 
restore the pointers to their original values.



