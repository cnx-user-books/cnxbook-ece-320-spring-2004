%AUTHOR: Daniel Sachs
The core file also supports the serial port installed on the
DSP320C549 DSP. The serial port on the EVM is attached to COM2 on
the PC.  Before jumping to your code, the core file initializes
the EVM's serial port to 38,400 bits per second with no parity, 1 stop bit
and 8 data bits. It then accepts characters received from the PC by
the UART (Universal Asynchonous Receiver/Transmitter) and buffers them
in memory until your code retrieves them. It also can accept a block of
bytes to transmit and send them to the UART in sequence.

Two macros are used to control the serial port: \verb+READSER+ and
\verb+WRITSER+.  Both accept one parameter. \verb+READSER <n>+ reads up
to \verb+<n>+ characters from the serial input buffer (the data coming
from the PC), and places them in memory starting at *AR3. (AR3 is left
pointing one past the last memory location written.) The actual number
of characters read is left in AR1. If AR1 is zero, then no characters
were available in the input buffer.

\verb+WRITSER <n>+ adds \verb+<n>+ characters starting at *AR3 to the
serial output buffer - in other words, it queues them to be sent to the
PC. AR3 is left pointing one location after the last memory location read.

Note that READSER and WRITSER modify registers AR0, AR1, AR2, AR3, and
BK as well as the flag TC. Be sure you restore these registers after
calling \verb+READSER+ and \verb+WRITSER+ if you need them later in
your code.

Note also that the core file allows up to 126 characters to
be stored in the input and output buffers. No checks to protect against
buffer overflows are made, so do not allow the input and output buffers
to overflow. (The length of the buffers can be changed by changing
\verb+ser_rxlen+ and \verb+ser_txlen+ values in the \verb+core.asm+
file.) The buffers are 127 characters long; however, the code
cannot distinguish between a completely-full and completely-empty
buffer. Therefore, only 126 characters can be stored in the buffers.

% add buffer-query macros?

It is easy to check if the input or output buffers in memory are
empty. The input buffer can be checked by comparing the values stored in
the memory locations \verb+srx_head+ and \verb+srx_tail+; if both memory
locations hold the same value, the input buffer is empty. Likewise, the
output buffer can be checked by comparing the values stored in memory
locations \verb+stx_head+ and \verb+stx_tail+. The number of characters in
the buffer can be computed by subtracting the head pointer from the tail
pointer; add the length of the buffer (normally 127) if the resulting
distance is negative.

The following example shows the minimal amount of code necessary to
echo received data back through the serial port. It has been placed in
\verb+V:\54x\dsplib\+ as \verb+ser_echo.asm+.

\setlength{\baselineskip}{0.4cm}
\listinginput{1}{ser_echo.asm}
\setlength{\baselineskip}{0.5cm}

On Line 8, we tell \verb+READSER+ to receive into the location \verb+hold+
by setting AR3 to point at it. On Line 9, we call \verb+READSER 1+ to
read one serial byte into \verb+hold+; the byte is placed in the low-order
bits of the word and the high-order bits are zeroed. If a byte was read,
AR1 will be set to 1. AR1 is checked in Line 12; Line 13 branches back to
the top if no byte was read. Otherwise, we tell reset AR3 to \verb+hold+
(since \verb+READSER+ moved it), then call \verb+WRITSER+ to send the
word we received on Line 16. On Line 18, we branch back to the start to
receive another character.
