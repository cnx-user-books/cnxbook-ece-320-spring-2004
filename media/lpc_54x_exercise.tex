%AUTHOR: Swaroop Appadwedula
However, the sampling rate on the 6-channel 
DSP boards is fixed at $44.1kHz$, so decimation of $5$ may be used 
to achieve the sampling rate of $8.82kHz$.  

Using either the test vector thru code or the input-output thru code, 
compute the autocorrelation or 
autocovariance coefficients of $256$-sample blocks of input samples from 
the function generator for time shifts $l=0,1,\ldots,15$ (i.e. $P=15$) and 
display these on the oscilloscope with a trigger.  (You may zero out 
the other $240$ output samples to fill up the $256$ sample block).  
For computing the autocorrelation, 
you will have to use memory to record the last $15$ samples of the input 
due to the overlap between adjacent blocks.  
Compare the output on the oscilloscope with the results from \matlab.

The next step is to use a speech signal as the input to your system.  
Use the powered microphone as the input to the original
\verb+thru.asm+ code and adjust the gains in your system until 
the output has reasonable amplitude and does not saturate.  
Now, you will need to write code to determine 
the start of a speech signal, record a few seconds of speech and 
compute the autocorrelation or autocovariance 
coefficients.  
In order to make a fair comparison with your \matlab version (with
speech sampled at $8 kHz$) you will have to resample the input.
A decimation factor of $5$ on the DSP is suggested resulting in a 
a sampling rate of $8.82 kHz$ for your speech input.
The start of a speech signal can be determined by comparing 
the input to some noise threshold.  For recording large segments of 
speech, you may need to use external memory. Refer to the handout on 
external memory for more information.

Finally, incorporate your code which computes autocorrelation or 
autocovariance coefficients with the code which takes speech 
input and compare the results seen on the oscilloscope to those 
generated by \matlab.

\paragraph{Optional}
In order to implement the Levinson-Durbin algorithm, you will need to 
use integer division to do Step 1 of the algorithm.  
Refer to the 54x Applications Guide and the \verb+subc+ 
instruction for a routine that performs integer division.

