
%
%
% Module: lms_matlab_exercise
%
% Author: Michael Kramer
%
%

Simulate the system identification block diagram
shown in Figure \ref{fig: sys_id} on a sample
by sample basis; that is, because the FIR filter
changes at each sample, you must use a ``\verb+do+''
loop in \matlab rather than the \verb+conv+ or \verb+filter+
functions, which use the same filter coefficients for the
entire input sequence.  For the unknown system, use the
fourth order low-pass elliptical IIR filter designed for
Lab 2.

Use Gaussian random noise your input, which can
be generated in \matlab using the command ``\verb+randn+''.
Simulate the system with an initial adaptive FIR
of zeros, starting with an adaptive filter of
length 32, and a step-size of $0.02$.
From your simulation you should be able
to plot the error (or squared-error) as it
evolves over time, as well as the final
set of adapted coefficients.  (How do they compare
to the unknown system coefficients?)

With your simulation working, you will then want
to experiment with different step-sizes and
adaptive filter lengths.

